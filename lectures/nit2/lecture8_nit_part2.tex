
%external style tex file
%Karina's directory
%\documentclass[t]{beamer} % what skinner put in his
%\documentclass[10pt]{beamer} % what porter put in his

%\usepackage[scaled=0.92]{helvet} % Helvetica
%\usepackage[options]{package}
\usepackage{etex} %package allows you to call more packages

%\usepackage[colorlinks,citecolor=DeepPink4,linkcolor=DarkRed, urlcolor=DarkBlue]{hyperref}
\usepackage{hyperref}
\usepackage{subfig}
\usepackage[all]{xy}
\usepackage{pgf}
\usepackage{pgfplots}
\usepackage{tikz}
\usetikzlibrary{decorations.pathreplacing}
\usetikzlibrary{arrows}
%\usetikzlibrary{snakes}
%\usetikzlibrary{snakes}
%\usetikzlibrary{decorations.pathmorphing,decorations.pathreplacing,decorations.shapes}
\usepackage{color}
\usepackage{colortbl}
\usepackage{booktabs}
\usepackage{pifont}
\usepackage{amssymb}
\usepackage{amsfonts}
%\usepackage{amsmath}
\usepackage{mathtools}
\usepackage{geometry}

\usepackage{graphicx}
\usepackage{graphics}
\usepackage{epsfig}
\usepackage{epstopdf}
\usepackage[space]{grffile} % to allow space in filepath name for graphicspath
\usepackage[final]{pdfpages} % insert entire pages of PDFs as is

%\epstopdfsetup{outdir=logs/}
	
\usepackage{lscape}
\usepackage{rotating}
\usepackage{siunitx}
\usepackage{sidecap}
\usepackage{multicol}
\usepackage{multirow}
\usepackage{verbatim}
\usepackage{caption}
\usepackage{array}
\usepackage{pdfpages}
\usepackage{import}
\usepackage{soul}
%\usepackage{stata}
%\usepackage{arydshl}%\usepackage{mathptmx}
%\usepackage{fourier}

\useoutertheme{infolines} 
\definecolor{links}{HTML}{2A1B81}
\hypersetup{colorlinks,linkcolor=,urlcolor=links}

%\usetheme{default}
%\usetheme{UofA}
\usetheme{Berkeley}
%\usetheme{Rochester}
%\usecolortheme{whale}
%\DeclareCaptionFont{footnotesize}{\footnotesize}

%\DeclareGraphicsExtensions{.eps,.pdf,.png,.jpg} % decide what image format you prefer if multiple images w/ same name but different format

% get rid of navigation icons:
\setbeamertemplate{navigation symbols}{} 

%\pgfplotsset{compat=1.10}

%\graphicspath{{C:/Users/ozanj/Documents/Dropbox/job-market-2015-16/nyu/job-talk/logs}}


%\section[Introduction]{Introduction: Overview of research program}

\AtBeginSection[]{
	\begin{frame}
		\vfill
		\centering
		\begin{beamercolorbox}[sep=8pt,center,shadow=true,rounded=true]{title}
			\usebeamerfont{section title}\insertsection\par%
		\end{beamercolorbox}
		\vfill
	\end{frame}
}


\author[Ozan Jaquette]{Ozan Jaquette\\ \footnotesize{\href{mailto:ozanj@ucla.edu}{\textcolor{blue}{\texttt{ozanj@ucla.edu}}}}}

% \institute[UCLA]{%
% 	\includegraphics[width=1.5cm]{C:/Users/ozanj/Dropbox/econometrics-course/lectures/header-tex/uclaseal}\\%remove this line if no logo.
% 	\textcolor{cyan}{{\Large University of California, Los Angeles}}\\
% 	Higher Education \& Organizational Change
% }

%\date[Jan, 2015] % (optional, should be abbreviation of conference name)
%{January 20, 2016}
% - Either use conference name or its abbreviation.


% If you have a file called "university-logo-filename.xxx", where xxx
% is a graphic format that can be processed by latex or pdflatex,
% resp., then you can add a logo as follows:

% \pgfdeclareimage[height=0.5cm]{university-logo}{university-logo-filename}
% \logo{\pgfuseimage{university-logo}}

% Delete this, if you do not want the table of contents to pop up at
% the beginning of each subsection:
%\AtBeginSubsection[]
%{
%	\begin{frame}<beamer>{Outline}
%		\tableofcontents[currentsection,currentsubsection]
%	\end{frame}
%}

%\author[Ozan Jaquette]{Ozan Jaquette\\ \footnotesize{\href{mailto:ozanj@email.arizona.edu}{ozanj@email.arizona.edu}}}

\newcommand{\bi}{\begin{itemize}}
\newcommand{\ei}{\end{itemize}}

 
\documentclass[t]{beamer} % what skinner put in his
%\documentclass[10pt]{beamer} % what porter put in his

%\usepackage[scaled=0.92]{helvet} % Helvetica
%\usepackage[options]{package}
\usepackage{etex} %package allows you to call more packages

%\usepackage[colorlinks,citecolor=DeepPink4,linkcolor=DarkRed, urlcolor=DarkBlue]{hyperref}
\usepackage{hyperref}
\usepackage{subfig}
\usepackage[all]{xy}
\usepackage{pgf}
\usepackage{pgfplots}
\usepackage{tikz}
\usetikzlibrary{decorations.pathreplacing}
\usetikzlibrary{arrows}
%\usetikzlibrary{snakes}
%\usetikzlibrary{snakes}
%\usetikzlibrary{decorations.pathmorphing,decorations.pathreplacing,decorations.shapes}
\usepackage{color}
\usepackage{colortbl}
\usepackage{booktabs}
\usepackage{pifont}
\usepackage{amssymb}
\usepackage{amsfonts}
%\usepackage{amsmath}
\usepackage{mathtools}
\usepackage{geometry}

\usepackage{graphicx}
\usepackage{graphics}
\usepackage{epsfig}
\usepackage{epstopdf}
\usepackage[space]{grffile} % to allow space in filepath name for graphicspath
\usepackage[final]{pdfpages} % insert entire pages of PDFs as is

%\epstopdfsetup{outdir=logs/}
	
\usepackage{lscape}
\usepackage{rotating}
\usepackage{siunitx}
\usepackage{sidecap}
\usepackage{multicol}
\usepackage{multirow}
\usepackage{verbatim}
\usepackage{caption}
\usepackage{array}
\usepackage{pdfpages}
\usepackage{import}
\usepackage{soul}
%\usepackage{stata}
%\usepackage{arydshl}%\usepackage{mathptmx}
%\usepackage{fourier}

\useoutertheme{infolines} 
\definecolor{links}{HTML}{2A1B81}
\hypersetup{colorlinks,linkcolor=,urlcolor=links}

%\usetheme{default}
%\usetheme{UofA}
\usetheme{Berkeley}
%\usetheme{Rochester}
%\usecolortheme{whale}
%\DeclareCaptionFont{footnotesize}{\footnotesize}

%\DeclareGraphicsExtensions{.eps,.pdf,.png,.jpg} % decide what image format you prefer if multiple images w/ same name but different format

% get rid of navigation icons:
\setbeamertemplate{navigation symbols}{} 

%\pgfplotsset{compat=1.10}

%\graphicspath{{C:/Users/ozanj/Documents/Dropbox/job-market-2015-16/nyu/job-talk/logs}}


%\section[Introduction]{Introduction: Overview of research program}

\AtBeginSection[]{
	\begin{frame}
		\vfill
		\centering
		\begin{beamercolorbox}[sep=8pt,center,shadow=true,rounded=true]{title}
			\usebeamerfont{section title}\insertsection\par%
		\end{beamercolorbox}
		\vfill
	\end{frame}
}


\author[Ozan Jaquette]{Ozan Jaquette\\ \footnotesize{\href{mailto:ozanj@ucla.edu}{\textcolor{blue}{\texttt{ozanj@ucla.edu}}}}}

% \institute[UCLA]{%
% 	\includegraphics[width=1.5cm]{C:/Users/ozanj/Dropbox/econometrics-course/lectures/header-tex/uclaseal}\\%remove this line if no logo.
% 	\textcolor{cyan}{{\Large University of California, Los Angeles}}\\
% 	Higher Education \& Organizational Change
% }

%\date[Jan, 2015] % (optional, should be abbreviation of conference name)
%{January 20, 2016}
% - Either use conference name or its abbreviation.


% If you have a file called "university-logo-filename.xxx", where xxx
% is a graphic format that can be processed by latex or pdflatex,
% resp., then you can add a logo as follows:

% \pgfdeclareimage[height=0.5cm]{university-logo}{university-logo-filename}
% \logo{\pgfuseimage{university-logo}}

% Delete this, if you do not want the table of contents to pop up at
% the beginning of each subsection:
%\AtBeginSubsection[]
%{
%	\begin{frame}<beamer>{Outline}
%		\tableofcontents[currentsection,currentsubsection]
%	\end{frame}
%}

%\author[Ozan Jaquette]{Ozan Jaquette\\ \footnotesize{\href{mailto:ozanj@email.arizona.edu}{ozanj@email.arizona.edu}}}

\newcommand{\bi}{\begin{itemize}}
\newcommand{\ei}{\end{itemize}}

 

\usepackage{comment}


%\usepackage{multicol}

%\usepackage[utf8]{inputenc}
%\usepackage[T1]{fontenc}

%Ozan's directory- need to change the style tex file to reflect UCLA email, course number, logos, etc.
%\input{C:/Users/ozanj/Documents/Dropbox/hed612/lectures/beamer-style/hed612-lecture-style2}


%\title [Short Title]{Long Title}
\title[EDUC 250B, Week 8] {EDUC 250B, Week 8: New Institutional Theory, Part II}
\subtitle{Comparing/integrating old and new institutionalism}
\date{Feb 26, 2020}

\begin{document}

\begin{frame}
	\titlepage
\end{frame}

\begin{frame}{Lecture overview}

	\tableofcontents
\end{frame}

%\section[Stata commands]{Matching commands in Stata}

\section[New institutionalism]{New institutionalism}

\subsection[Review]{Core ideas of new institutionalism [SKIP]}

\begin{frame}{Core ideas of new institutionalism [SKIP]}{Meyer and Rowan (1977)}
	
	Reasons for formal structure (departments, positions, policies)
	\begin{enumerate}
		\item Coordinate/control activities to create product (efficiency)
		\item Maintain legitimacy vis-a-vis stakeholders by adopting institutionalized practices
	\end{enumerate}
	\vspace{2mm}
	Institutions and rationalized myths
	\begin{itemize}
		\item Institution: taken for granted idea about appropriate practice
		\item Rationalized myths: rationalize a practice (e.g., adopt ``student success center'') based on efficiency/effectiveness
	\end{itemize}
	\vspace{2mm}
	Adherence to institutions/institutionalized ideas
	\begin{itemize}
		\item Often, survival doesn't depend on technical performance but on appearing legitimate to external stakeholders
		\item Orgs that incorporate institutionalized ideas into formal structure evaluated by adherence to institutions
	\end{itemize}	

\end{frame}

\begin{frame}{Core ideas of new institutionalism [SKIP]}{Meyer and Rowan (1977)}
	

	Conflict between technical demands and institutional demands
	\begin{itemize}
		\item Symbolic/ceremonial adoption of institutionalized practices
		\item Substantive adooption of demands necessary for technical success
	\end{itemize}
	\vspace{2mm}
	Decoupling/ceremonial inspection
	\begin{itemize}
		\item Decoupling: avoid rigorous inspection of efficacy of practice; instead measure effort or inputs
		\item Evaluations is ceremonial; just to say you did it
	\end{itemize}	

\end{frame}

\begin{frame}{Core ideas of new institutionalism}{DiMaggio and Powell (1983)}
	
	Isomorphism
	\begin{itemize}
		\item Process by which orgs adopt same structures and practices
		\item Technical: adopt practice because it is more efficient
		\item Institutional: Adopt taken for granted practices to be perceived as legitimate by stakeholders/orgs in the field
	\end{itemize}
	

	\vspace{2mm}
	Three broad forces/causes of institutional isomorphism
	\begin{enumerate}
		\item Coercive: pressure from resource provider (RDT)
		\item Mimetic: when uncertain, do what most/``leading'' orgs do
		\item Normative: professionalized fields socialize members to institutionalized ideas and members spread to orgs
	\end{enumerate}
	\vspace{2mm}
	Muddy waters
	\begin{itemize}
		\item Three forces not mutually exclusive (e.g., coercive isomorphism within professional fields)
		\item Change over time in which force most influential
	\end{itemize}
	
\end{frame}

\begin{frame}{Core ideas of new institutionalism}{DiMaggio and Powell (1983)}

	Fields and populations
	\begin{itemize}
		\item Organizational field
		\begin{itemize}
			\item All the stakeholders/players (e.g., suppliers, producers, consultants, regulators) in an  industry
		\end{itemize}
		\item Organizational population
		\begin{itemize}
			\item Set of orgs in a field with similar purpose, structure
		\end{itemize}
		\item Institutionalized ideas diffuse through field and orgs in a population adopt the same practices
	\end{itemize}	
	\vspace{2mm}
	Structuration: from heterogeneity to homogeneity over time
	\begin{itemize}
		\item When new fields/populations emerge, orgs exhibit diversity in structure, practices (e.g., ``junior colleges'')
		\item Repeated interaction, creation of associations, regulations cause institutionalized ideas to emerge, diversity declines
		\item Example: privacy policy by tech firms
	\end{itemize}
\end{frame}

\begin{comment}

\begin{frame}{Core ideas of new institutionalism}{Disagreements between Meyer/Rowan and DiMaggio/Powell}

	Meyer and Rowan (1977)
	\begin{itemize}
		\item highlights symbolic adoption to signal legitimacy to external environment; but technical core differs across orgs
	\end{itemize}
	\vspace{2mm}	
	DiMaggio and Powell (1983)
	\begin{itemize}
		\item (For mimetic and normative isomorphism) implicit assumption that adoption is substantive and internalized; we are all believers and orgs in a population really are the same
	\end{itemize}

\end{frame}

\end{comment}

\subsection{Early findings}

\begin{frame}{Early empirical research (the 1980s)}

	Definition of ``institution''
	\begin{itemize}
		\item taken for granted ideas about what to do and how to do it (e.g., must use a search firm to hire a new dean)
		\item Also, taken for granted ideas about meaning of thing/concept (e.g., ``property'' as an institution)

	\end{itemize}
	
	\vspace{2mm}
	Early research focused on \textbf{effects} of institutions
	\begin{itemize}
		\item Focus on diffusion of practices (e.g., chief diversity officer)
		\item Early adoption often for technical efficiency reasons, later adoption for legitimacy (Tolbert \& Zucker, 1983)
		\item Org response to coercive pressure often symbolic
	\end{itemize}
	\vspace{2mm}
	Usefulness for scholarship on equity (Meyer \& Rowan)
	\begin{itemize}
		\item Don't assume policy/practice adopted to solve problem
		\item Instead of studying ``best practices'' or if student success center is ``effective,'' study why it has no budget/staff
		%\item Adoption not always symbolic; examine budget, staff, powers (e.g., UCLA Equity Diversity Inclusion)
	\end{itemize}
\end{frame}

\begin{frame}{Criticisms in the late 1980s/early 1990s of new institutionalism}

	\begin{itemize}
		\item New institutionalism de-emphasizes agency
		\begin{itemize}
			\item Institutions are things that exist outside of orgs, created by big macro forces, but control behavior of orgs and people
		\end{itemize}	
		\item Researchers focus on effect of institutions; ignore how institutions form, persist, change, die
		\item Ignores power dynamics in the creation of institutions (who benefits)
		\item Is rejection of technical efficiency realistic?
		\begin{itemize}
			\item Dimmagio \& Powell (1983) argue that orgs adopt institutionalized practices, even when they are inefficient, at odds w/ technical environment
			\item When external environment changes, can an institutions at odds w/ new environment survive (e.g., liberal arts colleges only offer lib arts majors)?
		\end{itemize}
		\item Are we dupes mindlessly adopting institutions or are we savvy tricksters?
	\end{itemize}
\end{frame}

\subsection{New topics}

\begin{frame}{New institutionalism responds to critics by taking on new topics}

	\textbf{Institutionalization}: process by which institutions emerge
	\begin{itemize}
		\item \textit{Institutionalization projects}, \textit{institutional entrepreneurs}
		\begin{itemize}
			%\item Founders of American Assoc of Jr Colleges institutionalized the ``vocational'' mission (Brint \& Karabel)
			\item Focus on agency, conflict (e.g., Brint \& Karabel, 1989)
		\end{itemize}
	\end{itemize}		
	\vspace{2mm}
	\textbf{Deinstitutionalization}: process by which institutions die
	\begin{itemize}
		\item e.g., death of the conglomerate firm (Davis et al., 1994)
		\item Often happens when external environment changes and new institutions emerge that are more fit w/ environment
		\item These new institutions often created by ``illegitimate actors'' (e.g., hostile takeovers) (Hirsch, 1986)
	\end{itemize}
	\vspace{2mm}
	
	\textbf{Institutional maintenance/change}: Process by which existing institutions are maintained/changed
	\begin{itemize}
		\item \textit{Theorizing change} (Greenwood et al., 2002)
		\item Legitimacy of SAT exam challenged/maintained
	\end{itemize}


\end{frame}

\begin{frame}{Core finding from second wave of new institutionalism scholarship}{Market forces trump legitimacy (the business model has to work)}

	Early proponents posited that once institutionalized, an institution would persist even in the face of adverse external environment/market forces
	\begin{itemize}
		\item Turns out, this is usually not true!
		\item Once a new practice turns profitable, old practices are replaced (Davis, 2005); and vice-versa
	\end{itemize}
	\vspace{2mm}
	
	Examples:
	\begin{itemize}
		\item Death of conglomerate firm (Davis et al., 1994)
		\item End of lifetime employment in Japan (Ahmadjian \& Robinson, 2001)
		\item Liberal arts colleges offering business degrees (Kraatz \& Zajac, 1996)
	\end{itemize}
	
\end{frame}

\section[Old institutionalism]{Old institutionalism}

\subsection{Motivation}

\begin{frame}{Motivating the resurgence of ``old'' institutionalism}

	Continued dissatisfaction with new institutionalism
	\begin{itemize}
		%\item Focuses on creation and consequences of ``institutions,'' defined as macro forces that discipline organizations
		\item Ignores local power dynamics and local environment
		\item The theory not sufficient for analyzing dynamics at individual organizations (why stuff happens, how did this process play out)
		\begin{itemize}
			\item e.g., why did UCLA create the Office of Equity, Diversity, and Inclusion and what changes happened as a result?
		\end{itemize}
	\end{itemize}
	\vspace{2mm}
	Authors associated with old institutionalism
	\begin{itemize}
		\item Phillip Selznick; Burton Clark; Brint and Karabel 
		\item scholars of professions that utilize a ``conflict'' approach (e.g., Collins, 1979; Larson, 1977)
		\item Not much talk about ``old'' institutionalism until ``new'' institutionalism came around
	\end{itemize}
	\vspace{2mm}
	%Old institution has many similarities with garbage can theory and resource dependence theory
\end{frame}

\begin{frame}{Before diving into details of old institutionalism}{A broad comparison of new and old institutionalism}

	New institutionalism (Meyer/Rowan and DiMaggio/Powell)
	\begin{itemize}
		\item institutions: macro external forces, exert control on orgs
		\item Institutionalization happens at the field-level; an institution is some idea/practices that gets taken-for-granted for population of orgs within a field
		\item Research focus: populations of organizations in a field		
	\end{itemize}	
	\vspace{2mm}	
	Old institutionalism (e.g., Phillip Selznick; Karabel)		
	\begin{itemize}
		\item Focus on internal org dynamics (e.g., change in leadership) and interactions with local external environment (e.g., Alumni, local business leaders)
		\item Institutionalization happens at org-level; an institution is idea/practice that becomes widely accepted at org-level
		\item Research focus: individual org (e.g., Karabel, 1984); or particular group(s) of actors (e.g., Brint \& Karabel, 1989)
		
	\end{itemize}
	
\end{frame}


\begin{comment}
\begin{frame}{Core ideas of old institutionalism}

	Core concern of old institutionalism (Selznick)
	\begin{itemize}
		\item Orgs play important role in promoting societal values (e.g., democracy, equality of opportunity)
		\item Org upholds societal values only if org has mission/values consistent with broader societal values 
		\item Selznick concerned with orgs institutionalizing (at org-level) values/practices that promoted societal ideals
	\end{itemize}
	\vspace{2mm}
	The problem: org values/purpose are ``precarious''
	\begin{itemize}
		\item Change in leadership or org structure may displace values
		\item Coalitions composed of internal/external actors may seek to shift purpose of the org to serve their own interests
		\item Practices that garner resources, prestige may conflict with ideals the org supposed to uphold
	\end{itemize}
\end{frame}

\end{comment}

\begin{frame}{Core ideas of old institutionalism}

	Institutions are adhered to because:
	\begin{itemize}
		\item new institutionalism (mimetic): alternatives unthinkable (of course a university should have a Dean of Students)
		\item old institutionalism: focus on power, interest groups; adherence is result of conflict
	\end{itemize}
	%Actors and factors that old institutionalism tends to focus on
	\vspace{2mm}	
	Forces of (de)institutionalization in old institutionalism
	\begin{itemize}
		\item Long-standing org leaders are stewards of org values
		\begin{itemize}
			\item changes in org leaders or members can shift org values
		\end{itemize}
		\item Org structure upholds values; change in org structure can displace values by making offices with different priorities more powerful (e.g., enrollment mgt)
		\item Org values diverted by coalitions with different values
		\item Actors in local environment affect org mission (e.g., alum)
		\begin{itemize}
			\item In 1920s local elite Protestants pressured Harvard to adopt ``character'' admissions criteria to keep out Jews (Karabel)
		\end{itemize}
		
	\end{itemize}

\end{frame}

\subsection{Integration}

\begin{frame}{Using ``old institutionalism'' to study institutionalization of macro (``new'') institutions}

	Empirical applications of old institutionalism often show that ``macro'' institutions emerge from local-level actors
	
	\begin{itemize}
		\item this behavior/interactions better studied with tools of old institutionalism rather than new institutionalism
	\end{itemize}

	\vspace{2mm}	
	Examples of studies
	\begin{itemize}
		\item Brint and Karabel (1989): ``vocational'' mission of community colleges created by the men who founded the American Association of Junior Colleges
		\item Karabel (2005): admissions criteria utilized today by selective colleges was developed by Ivy League orgs trying to protect their interests and interests of local stakeholders
		\item Hirsch (1986): hostile corporate takeover (institutionalized in the 1980s), was created in 1970s by Jewish bankers excluded from firms and deals dominated by Protestants 

	\end{itemize}

\end{frame}

\begin{frame}{``Neo'' institutional theory: Incorporating both old and new institutionalism in empirical research}

	Institutional fields and mechanisms (Davis \& Marquis, 2005)
	\begin{itemize}
		\item First, choose your research question
		\item Sketch ``institutional field'': the orgs, external forces, regulations, etc. relevant to the thing you are studying
		\item The theories you choose: (a) point to certain interactions and actors as influencing the thing you are studying and  (b) explain why/how these interactions and actors are influential (these explanations are the mechanisms)
		\begin{itemize}
			%\item e.g., normative isomorphism suggests that impetus to change "disability resource center" to "center for accessible education emerges from professional associations
			\item e.g., normative isomorphism says professional associations influence change to ``center for accessible'' education
		\end{itemize}
		
	\end{itemize}	
	\vspace{2mm}
	Using old and new (better for case studies than large N)
	\begin{itemize}
		\item Introduce new institutionalism concepts to highlight macro forces affecting your research focus
		\item Introduce old institutionalism concepts to highlight local forces (within and near org) affecting your research focus
	\end{itemize}
\end{frame}

\begin{frame}{Critique old/new/neo flavors of institutionalism}

	\textbf{Old}: local internal/external org dynamics that affect which practices get institutionalized at org-level
	\begin{itemize}
		\item critique: ignores broad forces that affect orgs
	\end{itemize}		
	\textbf{New}: Macro forces affect which practices adopted by all orgs
	\begin{itemize}
		\item critique: minimizes agency of individuals/groups
	\end{itemize}
	\textbf{Neo}: draw from ``old' and ``new'' institutionalism
	\begin{itemize}
		\item macro institutions emerge from particular actors (ASHE)
		\item Org adopts macro institution in response to local pressure (GSEIS adopts EDI dean in response to local concerns)
	\end{itemize}


	\vspace{3mm}
	Ignorance shared by all flavors of institutionalism
	\begin{itemize}
		\item Macro structures/institutions systematically privilege dominant (white) groups 
		\item Macro institutions build on established macro institutions that were designed to benefit white men
	\end{itemize}
	
\end{frame}



\section{References}

\begin{frame}{(Partial) reference list}
	
	\begin{itemize}
	{\tiny
		\item Ahmadjian, C. L., \& Robinson, P. (2001). Safety in numbers: Downsizing and the deinstitutionalization of permanent employment in Japan. Administrative Science Quarterly, 46(4)%, 622-65
		\item Brint, S. G., \& Karabel, J. (1989). The diverted dream: community colleges and the promise of educational opportunity in America, 1900-1985. New York: Oxford University Press.
		\item Clark, B. R. (1956). Organizational adaptation and precarious values: a case study. American Sociological Review, 21(3), 327-336.
		\item Collins, R. (1979). The credential society: An historical sociology of education and stratification. New York: Academic Press.
		\item Davis, G. F., Diekmann, K., \& Tinsley, C. (1994). The decline and fall of the conglomerate firm in the 1980s: The deinstitutionalization of an organizational Form. American Sociological Review, 59(4), 547-570. 
		\item Davis, G. F. (2005). Firms and environments. In N. J. Smelser \& R. Swedberg (Eds.), The handbook of economic sociology (pp. 478-502). New York: Russell Sage Foundation.
		\item Davis, G. F., \& Marquis, C. (2005). Prospects for organization theory in the early twenty-first century: Institutional fields and mechanisms. Organization Science, 16(4), 332-343. 
		%\item Greenwood, R., Suddaby, R., \& Hinings, C. R. (2002). Theorizing change: The role of professional associations in the transformation of institutionalized fields. Academy of Management Journal, 45(1), 58-80. 
		\item Hirsch, P. M. (1986). From ambushes to golden parachutes: corporate takeovers as an instance of cultural framing and institutional integration. American Journal of Sociology, 91(4), 800-837. 
		\item Karabel, J. (1984). Status-group struggle, organizational interests, and the limits of institutional autonomy: The transformation of Harvard, Yale, and Princeton, 1918-1940. Theory and Society, 13(1), 1-40. 
		\item Karabel, J. (2005). The chosen: The hidden history of admission and exclusion at Harvard, Yale, and Princeton. Boston: Houghton Mifflin Co.
		\item Kraatz, M. S., \& Zajac, E. J. (1996). Exploring the limits of the new institutionalism: The causes and consequences of illegitimate organizational change. American Sociological Review, 61(5)%, 812-836.
		\item Larson, M. S. (1977). The rise of professionalism: a sociological analysis. Berkeley: University of California Press.
		\item Selznick, P. (1949). TVA and the grass roots: a study in the sociology of formal organization. Berkeley,: Univ. of California Press.
		\item Tolbert, P. S., \& Zucker, L. G. (1983). Institutional sources of change in the formal structure of organizations: The diffusion of civil-service reform, 1880-1935. Administrative Science Quarterly, 28(1), 22-39. 
		
	}		
	\end{itemize}

\end{frame}

\end{document}

